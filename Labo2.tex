\documentclass[a4paper,11pt]{article}

% Packages
\usepackage[table,x11names]{xcolor}
\usepackage{placeins}
\usepackage{textcomp}
\usepackage[utf8]{inputenc}
\usepackage[T1]{fontenc}
\usepackage{graphicx}
\usepackage{grffile}
\usepackage{longtable}
\usepackage{wrapfig}
\usepackage{rotating}
\usepackage[normalem]{ulem}
\usepackage{amsmath}
\usepackage{amssymb}
\usepackage{capt-of}
\usepackage[colorlinks=true,linkcolor=blue]{hyperref}
%%%
\usepackage[square,numbers]{natbib}
\bibliographystyle{abbrvnat}
%%%
\usepackage[french]{babel}
\usepackage[final]{pdfpages}
\usepackage[hmargin=2.75cm,vmargin=2.5cm]{geometry}
\usepackage{lastpage}
\usepackage{fancyhdr}
\usepackage{xfrac}
\usepackage{cancel}
\usepackage{marvosym}
\usepackage{diagbox}
\usepackage{multirow}
\usepackage[nottoc]{tocbibind}

%%%%%%%%%%%%%%%%%%%%%%%%%%%%%%%%%%%%%%%%%%%%%%
\usepackage{enumitem}

\definecolor{light-gray}{gray}{0.85}

\newcommand{\Ccancel}[2][black]{\renewcommand\CancelColor{\color{light-gray}}\cancel{#2}}

%%%%%%%%%%%%%%%%%%% Valeurs à modifier !!! %%%%%%%%%%%%%%%%%%
\def \datered {29 mars 2022} % date de reddition
\def \dated {04.04.2022} % du dd.mm.aa
\def \datef {25.04.2022} % au dd.mm.aa 
\def \nom {Wiggins} % Votre nom
\def \prenom {\ Eliott} % Votre prénom
\def \labo {Gestion créative des droits d’accès et
	permissions dans Unix/Linux} % Intitulé du laboratoire

%%%%%%%%%%%%%%%%%%%%%%%%%%%%%%%%%%%%%%%%%%%%%%%%%%%%%%%%%%%%%

% Quelques modifications
\setlength{\parskip}{1em} % Pour l'espace entre les paragraphes
\frenchbsetup{StandardItemLabels=true, CompactItemize=false, ReduceListSpacing=true} % Pour les listes
\def\UrlBreaks{\do\/\do-}  % Pour la présentation des longues URL dans la bibliographie
\Urlmuskip=0mu plus 4mu    % idem
\graphicspath{{figures/}}  % Pour directement chercher les figures dans le
% fichier figures

% Entête et bas de page
\fancyhead[L]{\nom \ \prenom}
\fancyhead[R]{Haute école de gestion Genève - \bsc{Informatique}}
\fancyfoot[L]{63-22 Labo 2}
\fancyfoot[C]{\thepage /  \pageref{LastPage}}
\fancyfoot[R]{\today}
\fancypagestyle{titlestyle}{
	\fancyhf{}
	\fancyfoot[C]{\thepage /  \pageref{LastPage}}
	\renewcommand{\headrulewidth}{0pt}
}
\setlength{\headheight}{13.6pt}

% Pour l'hypersetup (données/paramètres du fichier pdf)
\hypersetup{
	pdfauthor={\nom \prenom},
	pdftitle={\labo},
	pdfkeywords={},
	pdfsubject={\labo},
	pdfcreator={},
	pdflang={French},
	colorlinks=true,
	linkcolor=blue,
	citecolor=blue,
	urlcolor=blue}

%%%%%%%%%%%%%%%%%%%%%%%%%%%%to do list %%%%%%%%%%%%%%%%%%%%%%%%%
\newlist{todolist}{itemize}{2}
\setlist[todolist]{label=$\square$}


%%%%%%%%%%%%%%%%%%%%%%% Début du document %%%%%%%%%%%%%%%%%%
\begin{document}
	
	\renewcommand{\tablename}{Tableau} % Tableau au lieu de Table
	\renewcommand{\figurename}{Figure} % Graphique au lieu de Figure
	\setlength{\parindent}{0pt}
	

	%%%%%%% Page de titre (optionnel)
	\begin{titlepage}
		\thispagestyle{titlestyle}
		\newcommand{\HRule}{\rule{\linewidth}{0.2mm}} % Defines a new command for the horizontal lines, change thickness here
		\begin{center} % Center everything on the page
		{\Large Sécurité informatique - 63-22} \\[1.5cm]
		\HRule \\[0.6cm]
		
		{\huge \bfseries \labo}\\[0.4cm] % Title of your  document
		\HRule \\[1.5cm]
		{\Large \prenom \ \bsc{\nom}}\\[0.2cm] % Your name
		{\large \today}\\[2cm] % Date, change the \today to a set date if you want to be precise
		\vfill % Fill the rest of the page with whitespace
		\setcounter{page}{1}
		\end{center}
	\end{titlepage}
	
	
	%%%%%%% Table des matières (obligatoire)
	\pagestyle{fancy}
	\setcounter{page}{2} % Changer 2 par 1 s'il n'y a pas de page de titre
	\tableofcontents
	\newpage
	
	
	%%%%%%%%% Il faut commencer à rédiger le rapport ici !!!
	
	
	
	\section{Introduction }
	Cette correction est basée sur une version plus facile du Labo 2 de linux/unix pour le 63-22. Si la connexion par ssh peut poser problème, il est recommandé de faire directement les manipulations dans le terminal
	
	Attention : IL PEUT Y AVOIR DES ERREURS DANS LA CORRECTION
	\section{SSH}
	J'utilise une machine virtuelle ubuntu server (donc pas d'interface graphique) et le WSL pour se connecter. 
	
	Le but de ssh (secure shell) est d'utiliser la ligne de commande à distance du "vrai serveur" (même principe que telnet mais le protocole ssh est sécurisé).
	
	Sur la machine virtuelle ubuntu, on effectue la commande \textit{ip a} (sans -) afin de voir quelle adresse ip cette machine possède. \textbf{Attention il faut avoir activé une interface bridge dans virtual box, une fois la machine virtuelle éteine : configuration, mode expert, réseau, ajout d'un 2e adaptateur réseau en mode "bridge" ou "accès par pont" en français}
	
	La machine virtuelle ubuntu server va me demander un login, je met le login configuré dans virtual box (de base c'est "vboxuser" et mdp:"changeme") Ici mon login est "eliott" et le mdp: "citron"
	\begin{equation*}
		\text{ubuntu \quad  login} : \quad eliott
	\end{equation*}
	\begin{equation*}
		\text{Password :} \quad citron
	\end{equation*}
	\newline
	\textit{Attention des fois il y a des problèmes de clavier pour les caractères types / \% etc.  } \newline
	On laisse la machine allumée et on se dirige vers un terminal de notre machine, soit le cmd windows, soit le WSL (Windows SubLinux System). J'utilise ici le WSL.
	
	 
	\section{Opérations}
	1. Créez un répertoire (dans votre répertoire personnel) appelé Data.
	
	\textit{mkdir Data}
	
	\section{}
	
	\begin{equation}
		\text{S1}>\textbf{enable}
	\end{equation}
	\begin{equation}
		\text{S1} \#  \quad\textbf{delete vlan.dat}
	\end{equation}
	\begin{equation}
		\text{S1} \# \quad \textbf{write erase}
	\end{equation}
	\begin{equation}
		\text{S1} \# \quad \textbf{reload}
	\end{equation}

\subsection{}
\begin{equation}
	\text{Switch} \# \quad \textbf{configure terminal}
\end{equation}
\begin{equation}
	\text{Switch(config)} \# \quad \textbf{hostname \textit{Switch-1}}
\end{equation}
\begin{equation}
	\text{Switch-1(config)} \# \quad \textbf{service password-encryption}
\end{equation}
\begin{equation}
	\text{Switch-1(config)} \# \quad \textbf{enable secret \textit{class}}
\end{equation}
\begin{equation}
	\text{Switch-1(config)} \# \quad \textbf{no ip domain-lookup}
\end{equation}
\begin{equation}
	\text{Switch-1(config)} \# \quad \textbf{ip default-gateway} \quad  192.168.1.1
\end{equation}
\subsection{}

\subsubsection{}


\begin{equation}
	\text{Switch-1(config)} \# \quad \textbf{line console 0}
\end{equation}
\begin{equation}
	\text{Switch-1(config-line)} \# \quad \textbf{ password \textit{cisco}}
\end{equation}
\begin{equation}
	\text{Switch-1(config-line)} \# \quad \textbf{login}
\end{equation}
\begin{equation}
	\text{Switch-1(config-line)} \# \quad \textbf{exit}
\end{equation}

\subsubsection{}
\begin{equation}
	\text{Switch-1(config)} \# \quad \textbf{line vty 0 15}
\end{equation}
\begin{equation}
	\text{Switch-1(config-line)} \# \quad \textbf{ password \textit{cisco}}
\end{equation}
\begin{equation}
	\text{Switch-1(config-line)} \# \quad \textbf{login}
\end{equation}
\begin{equation}
	\text{Switch-1(config-line)} \# \quad \textbf{exit}
\end{equation}

	%%%%%
	
	
\section{}
\begin{equation}
	\text{Switch-1(config)} \# \quad \textbf{interface vlan \textit{1}}
\end{equation}
\begin{equation}
	\text{Switch-1(config-if)} \# \quad \textbf{description \textit{interface de gestion}}
\end{equation}	
\begin{equation}
	\text{Switch-1(config-if)} \# \quad \textbf{ip address} \quad 192.168.1.6 \quad 255.255.255.0
\end{equation}	
\begin{equation}
	\text{Switch-1(config-if)} \# \quad \textbf{no shutdown}
\end{equation}	
	%%%%%
	\section{Vérification}
\begin{equation}
	\text{Switch-1(config)} \# \quad \textbf{do show running-config}
\end{equation}
ou alors : 
\begin{equation}
	\text{Switch-1} \# \quad \textbf{show running-config}
\end{equation}
	%%%%%
	%\section{Références}
	
	%\section*{Annexe}
	
	
	
	
\end{document}

%%%%%%%%%%%%%%%%%%%%%%%%% Fin du document %%%%%%%%%%%%%%%%%%%%%%%%%%
